%%
%% The first command in your LaTeX source must be the \documentclass command.
\documentclass[acmsmall,screen,10pt,nonacm]{acmart}
% \renewcommand\footnotetextcopyrightpermission[1]{}
% \settopmatter{printacmref=false}

%% \BibTeX command to typeset BibTeX logo in the docs
\AtBeginDocument{%
  \providecommand\BibTeX{{%
    \normalfont B\kern-0.5em{\scshape i\kern-0.25em b}\kern-0.8em\TeX}}}

% \usepackage[shortcuts]{extdash}
% \usepackage{multirow}
% \usepackage{graphicx}
\usepackage{xcolor}
\usepackage{soul}
\usepackage{listings}
\usepackage{courier}

\let\defaulttexttt\texttt
\renewcommand{\texttt}[1]{\defaulttexttt{\small{#1}}}

\lstset{basicstyle=\footnotesize\ttfamily,breaklines=true,captionpos=b}
\lstset{framextopmargin=50pt,frame=bottomline}

%% Rights management information.  This information is sent to you
%% when you complete the rights form.  These commands have SAMPLE
%% values in them; it is your responsibility as an author to replace
%% the commands and values with those provided to you when you
%% complete the rights form.
% \setcopyright{acmcopyright}
% \copyrightyear{2018}
% \acmYear{2018}
% \acmDOI{10.1145/1122445.1122456}

%% These commands are for a PROCEEDINGS abstract or paper.
\acmConference[Woodstock '18]{Woodstock '18: ACM Symposium on Neural
  Gaze Detection}{June 03--05, 2018}{Woodstock, NY}
\acmBooktitle{Woodstock '18: ACM Symposium on Neural Gaze Detection,
  June 03--05, 2018, Woodstock, NY}
\acmPrice{15.00}
\acmISBN{978-1-4503-XXXX-X/18/06}


%%
%% Submission ID.
%% Use this when submitting an article to a sponsored event. You'll
%% receive a unique submission ID from the organizers
%% of the event, and this ID should be used as the parameter to this command.
%%\acmSubmissionID{123-A56-BU3}

%%
%% The majority of ACM publications use numbered citations and
%% references.  The command \citestyle{authoryear} switches to the
%% "author year" style.
%%
%% If you are preparing content for an event
%% sponsored by ACM SIGGRAPH, you must use the "author year" style of
%% citations and references.
%% Uncommenting
%% the next command will enable that style.
%%\citestyle{acmauthoryear}

%%
%% end of the preamble, start of the body of the document source.
\begin{document}

\definecolor{UoC}{RGB}{137 33 27}
\begin{center}
\includegraphics[scale=0.2]{figures/UoC_logo.png}\\
\textcolor{UoC}{\textbf{UNIVERSITY OF CRETE}}
\vspace{3em}
\end{center}

%%
%% The "title" command has an optional parameter,
%% allowing the author to define a "short title" to be used in page headers.
\title{Integration of Component A in System B}
\subtitle{Thesis submitted in partial fulfilment of the requirements for the degree of Computer Science}

\thanks{\small{This thesis was supervised by Prof. John Doe, co-supervised by Jane Doe, PhD, Institute of Computer Science, FORTH, and submitted in July 2021}}

%%
%% The "author" command and its associated commands are used to define
%% the authors and their affiliations.
%% Of note is the shared affiliation of the first two authors, and the
%% "authornote" and "authornotemark" commands
%% used to denote shared contribution to the research.
\author{Aulus Agerius}
\affiliation{%
  \department{Computer Science Department}
  \institution{University of Crete}
  \city{Heraklion}
  \country{Greece}
}
\email{csd0000@csd.uoc.gr}

%%
%% By default, the full list of authors will be used in the page
%% headers. Often, this list is too long, and will overlap
%% other information printed in the page headers. This command allows
%% the author to define a more concise list
%% of authors' names for this purpose.
\renewcommand{\shortauthors}{A. Agerius}

%%
%% The abstract is a short summary of the work to be presented in the
%% article.
\begin{abstract}
Lorem ipsum dolor sit amet, consectetur adipiscing elit. Nam congue lacus eget quam bibendum pulvinar. Pellentesque sit amet convallis tortor. Vivamus nisi justo, volutpat et quam quis, semper fermentum lectus. Fusce et laoreet erat, at ultricies nibh. Donec malesuada sapien vitae ligula scelerisque tempor. Morbi maximus erat non nulla blandit pulvinar. Vivamus fringilla ipsum pulvinar mi molestie, vel fermentum elit bibendum. Proin turpis est, faucibus vehicula erat a, semper tempus nisi. Praesent ullamcorper mollis porttitor. Suspendisse varius quam arcu, id aliquet turpis rhoncus in. Suspendisse ac lobortis nibh. Curabitur libero ligula, dapibus ut tristique at, porta eu ante. Fusce scelerisque augue lorem, et suscipit felis facilisis nec. Maecenas ullamcorper quam eros, sed euismod neque ullamcorper quis. Quisque consequat eleifend risus posuere laoreet. Nullam tempus velit eu mauris rhoncus, a porta purus mattis.

\end{abstract}

%%
%% The code below is generated by the tool at http://dl.acm.org/ccs.cfm.
%% Please copy and paste the code instead of the example below.
%%
% \begin{CCSXML}
% <ccs2012>
%  <concept>
%   <concept_id>10010520.10010553.10010562</concept_id>
%   <concept_desc>Computer systems organization~Embedded systems</concept_desc>
%   <concept_significance>500</concept_significance>
%  </concept>
%  <concept>
%   <concept_id>10010520.10010575.10010755</concept_id>
%   <concept_desc>Computer systems organization~Redundancy</concept_desc>
%   <concept_significance>300</concept_significance>
%  </concept>
%  <concept>
%   <concept_id>10010520.10010553.10010554</concept_id>
%   <concept_desc>Computer systems organization~Robotics</concept_desc>
%   <concept_significance>100</concept_significance>
%  </concept>
%  <concept>
%   <concept_id>10003033.10003083.10003095</concept_id>
%   <concept_desc>Networks~Network reliability</concept_desc>
%   <concept_significance>100</concept_significance>
%  </concept>
% </ccs2012>
% \end{CCSXML}

% \ccsdesc[500]{Computer systems organization~Embedded systems}
% \ccsdesc[300]{Computer systems organization~Redundancy}
% \ccsdesc{Computer systems organization~Robotics}
% \ccsdesc[100]{Networks~Network reliability}

%%
%% Keywords. The author(s) should pick words that accurately describe
%% the work being presented. Separate the keywords with commas.
% \keywords{datasets, neural networks, gaze detection, text tagging}

%%
%% This command processes the author and affiliation and title
%% information and builds the first part of the formatted document.
\maketitle

\section{Introduction}

Vivamus nec sapien a lorem finibus convallis ut vitae eros. Integer id augue nec ex malesuada scelerisque. In ante velit, imperdiet sed ex non, ultrices laoreet urna. Morbi ut purus neque. Donec vulputate in urna non viverra. Mauris a mi vel augue auctor semper. Nam mollis commodo enim at imperdiet. In hac habitasse platea dictumst. Orci varius natoque penatibus et magnis dis parturient montes, nascetur ridiculus mus. Integer scelerisque, sapien non tristique pharetra, purus purus tempor sem, quis ornare justo magna vitae nisl. Suspendisse potenti. Nulla et bibendum urna. Nam malesuada ac diam ut cursus \cite{lipsum}.

\section{Design and implementation}

Nunc neque quam, scelerisque vel justo non, viverra ornare dolor. Maecenas nisl ligula, efficitur eu justo at, gravida semper velit. Maecenas quis convallis velit, et elementum justo. Duis aliquet, risus eu posuere ornare, sem metus consequat neque, vehicula dignissim justo dolor id justo. Aenean accumsan pulvinar justo id fringilla. In ultrices felis tempus varius consequat. Mauris scelerisque massa pharetra tristique scelerisque. Vestibulum elit tellus, pellentesque eget orci sit amet, venenatis hendrerit arcu. Pellentesque eu justo mauris. Cras vel dapibus ipsum. Praesent nec malesuada urna, sit amet lacinia velit. Vivamus pharetra volutpat hendrerit. Curabitur tristique ut justo id dignissim. Donec tempor gravida ultricies.

Sed ac pharetra quam. Praesent sed fermentum ante, et congue arcu. Suspendisse potenti. Vestibulum dolor enim, faucibus vitae blandit in, hendrerit eu dui. Sed fringilla sed sem vel dapibus. Cras pellentesque neque ligula, ac fermentum enim posuere at. In hac habitasse platea dictumst. Cras vitae tellus at orci imperdiet euismod. Ut ultrices sollicitudin efficitur. Quisque porttitor eget nisi non dapibus. Aenean molestie velit quis efficitur venenatis. Morbi ac purus ac tortor maximus fringilla quis ut arcu. Nulla sollicitudin id velit congue laoreet.

\section{Conclusion}

Praesent in aliquet metus. Nullam et tellus elit. In ornare dui et lectus luctus, vel interdum nisi rhoncus. Sed ultricies libero sed velit pretium, vel interdum dui varius. Nullam quis luctus risus. Donec nulla quam, suscipit eu scelerisque in, gravida at dolor. Duis feugiat, elit in consectetur egestas, neque libero vestibulum libero, vel congue dui dui a nunc. Morbi ut mattis neque. Nullam tempus nunc eget urna condimentum semper.

\begin{acks}
The author would like to thank Numerius Negidius for his valuable comments and suggestions during this work.
\end{acks}


\bibliographystyle{ACM-Reference-Format}
\bibliography{references}

\end{document}
